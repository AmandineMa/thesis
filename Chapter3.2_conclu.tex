\ifdefined\included
\else
\documentclass[a4paper,11pt,twoside]{StyleThese}
\include{formatAndDefs}
\sloppy
\begin{document}
	\setcounter{chapter}{3} %% Numéro du chapitre précédent ;)
	\dominitoc
	\faketableofcontents
	\fi

\clearpage
\chapter{Conclusion and Future work}
In this chapter, we have proposed a generic supervision component providing robot decision and control to the robotic architecture it is integrated to, for collaborative robots. The developed \acrfull{rja} have their basis into the joint action principles presented in Chapter~\ref{chapter:chap1}. It endows the robot with knowledge representation, perspective-taking and \acrshort{tom}, joint attention, monitoring, communication, shared plan management.  


We had a lot of ideas but implemented only a few of them. To build a supervision system to endow a robot with autonomy when performing collaborative tasks with a human is not an easy thing, especially when thinking to genericness and re-usability. The work presented in this chapter provides a basis for an even more complex system, handling contingencies, and with a more refined interaction session manager which could integrate a nice goal negotiation component. Moreover, once the robot performs quite well with one task and one human, why not add other humans and/or other tasks in parallel?

%limit : use of shared plans as these ones. what should be interesting would be to have plans with actions having conditions about what should be executed before it and not predecessors

%future work : to be interruptible to do another task with someone else when already performing a task, when it is possible and socially acceptable
% gestion de buts
% action monitoring : pouvoir reconnaitre plus de types d'action (idle, answering the phone etc and so distinguish the ones the human is doing for the task or other actions outside the task)

In Section~\ref{chap2:sec:control}, we presented the \acrshort{jahrvis} processes for decision-making and robot control. Chapter~\ref{chapter:chap3}, n the context of a direction-giving task, will feature the \acrlong{ism}, an early version of the \acrlong{rpm} and the \acrlong{aem}. Chapter~\ref{chapter:chap4}, as for it, will emphasize the \acrlong{rpm}, the \acrlong{hpm}, the \acrlong{aem}, the \acrlong{ham} and the \acrlong{cm}.

\ifdefined\included
\else
\bibliographystyle{acm}
\bibliography{These}
\end{document}
\fi